%===========================================================
%== Präambel ===============================================
%===========================================================
\documentclass[
    paper=a4,
    bibtotocnumbered,
    liststotocnumbered,
    oneside,
    12pt,
    listof=totoc,
    toc=chapterentrywithdots,
    listof=entryprefix,
]{scrreprt}

\usepackage[a4paper, left=2.5cm, right=2.5cm, top=1.25cm, bottom=1.25cm, includehead, includefoot]{geometry}


\usepackage{float}
\usepackage[T1]{fontenc}%Encoding
\usepackage[german]{babel}%German-specific commmands
%\usepackage[utf8]{inputenc} %Wird nicht benötigt bei XeLatex
\usepackage{fancyhdr}%% Kopfzeile u. Fußzeile
\usepackage{fontspec}%% Schriftart
\usepackage[onehalfspacing]{setspace}%-- Zeilenabstand 1,5
\usepackage[toc, nopostdot, acronyms]{glossaries}%% Glossar
\usepackage[intoc]{nomencl}%% Symbolverzeichnis
\usepackage{etoolbox}%Sorgt dafür, das Symbolverzeichnis im Header steht
%\usepackage{listings}
\usepackage[]{minted}
\usepackage[titles]{tocloft}
\usepackage{scrhack}
\usepackage{colortbl}%Einfärben von Tabellen-Zellen, Zeilen, Spalten
\usepackage[]{layout}
\usepackage[]{blindtext}
%\usepackage{showframe}% zum Anzeigen des Seitenlayouts
\usepackage[figurename={Abb.},tablename={Tab.}]{caption}%Anpassung der Namen für Abbildung und Tabelle
\usepackage[]{url}
\usepackage[]{hyperref}
\usepackage{chngcntr}%% Nummerierung der Abbildungen und Tabellen fortlaufend %%%%%%
\usepackage{tocbibind}

% Paragraph styles
\setlength{\parindent}{0cm}
\setlength{\parskip}{6pt}

%% Kopfzeile u. Fußzeile
\pagestyle{fancy}
\fancyfoot{}
\fancyfoot[R]{\thepage}
\fancyhead{}
\fancyhead[L]{\nouppercase{\leftmark}}
\fancypagestyle{plain}{\pagestyle{fancy}}


%-- Standardschriftart auf Times New Roman umstellen
\setmainfont[ Path = Fonts/,
BoldFont=timesbd.ttf,
ItalicFont=timesi.ttf,
BoldItalicFont=timesbi.ttf
]{times.ttf}

%-- Kapitelüberschriften auf Standardschriftart(Times New Roman) umstellen
\setkomafont{disposition}{\normalcolor\bfseries}

% Entfernt "Kapitel X" aus der Kopfzeile vor der Kapitelüberschrift
\renewcommand{\chaptermark}[1]{\markboth{\MakeUppercase{#1}}{}}
 
%Abstände nach Kapiteln 
\RedeclareSectionCommand[%
  beforeskip=0pt,
  afterskip=1\baselineskip plus .1\baselineskip minus .167\baselineskip
]{chapter}

%% Nummerierung der Abbildungen und Tabellen fortlaufend %%%%%%
\counterwithout{figure}{chapter}
\counterwithout{table}{chapter}

%% Rahmen in Textbreite um Bilder
\floatstyle{boxed}
\restylefloat{figure}
\restylefloat{table}
\restylefloat{listing}

%% Captions linksbündig orientieren
\captionsetup{justification=raggedright,singlelinecheck=false}

% Einbinden von Glossar u. Abkürzungsverzeichnis
\input{Verzeichnisse/Glossar.tex}
\input{Verzeichnisse/Abkürzungsverzeichnis.tex}
\makeglossaries

\begin{document}

%===========================================================
%== Titelseite =============================================
%===========================================================
\newgeometry{left=2.5cm, right=2.5cm, top=2.5cm, bottom=2.5cm}
\begin{titlepage}
    \begin{center}
    \includegraphics[width=\textwidth]{Images/logo}
        \begin{Large}
        Hochschule für angewandte Wissenschaften Coburg
        \\
        Fakultät Elektrotechnik und Informatik
        \par
        \end{Large}
        \vspace{2.0cm}
        
        \begin{Large}
            Studiengang: Informatik
            \par
        \end{Large}
        \vspace{1.5cm}
        
        \begin{Large}
            Exposé zur Bachelorarbeit
        \end{Large}
        \vspace{2.0cm}

        \begin{huge}
            \textbf{Vergleich von Apple Swift und Google Go} 
            \par
        \end{huge}        
        \vfill
        
        \begin{huge}
            Daniel Müller
        \end{huge}
        \vspace{2.0cm}
        
        \begin{large}
            %Abgabe der Arbeit: DD. Monat 2015
            
            Betreut durch:
            \\
            <Prof. Dr.> <Vorname> <Nachname>, Hochschule Coburg
            \\
            %<optional: Zweitgutachter: Prof. Dr. XXX, Hochschule Coburg>
            \par
        \end{large}
        
    \end{center}
\end{titlepage}
\restoregeometry
    
%===========================================================
%== Inhaltsverzeichnis =====================================
%===========================================================   
\begin{singlespace}
\renewcommand{\cftchapleader}{\cftdotfill{\cftdotsep}} % Punkte im Inhaltsverzeichnis bei Kapiteln
\tableofcontents
\end{singlespace}

% %===========================================================
% %== Abbildungsverzeichnis ==================================
% %===========================================================
% \newpage
% \renewcommand{\cftfigpresnum}{Abb. }
% \renewcommand{\cftfigaftersnum}{:}
% \setlength{\cftfignumwidth}{2cm}
% \setlength{\cftfigindent}{0cm}
% \listoffigures

% %===========================================================
% %== Tabellenverzeichnis ====================================
% %===========================================================
% \newpage
% \renewcommand{\cfttabpresnum}{Tab. }
% \renewcommand{\cfttabaftersnum}{:}
% \setlength{\cfttabnumwidth}{2cm}
% \setlength{\cfttabindent}{0cm}
% \listoftables

% %===========================================================
% %== Codeverzeichnis ========================================
% %===========================================================
% \newpage
% \renewcommand\listingscaption{Code}
% \renewcommand\listoflistingscaption{Codebeispielverzeichnis}
% \renewcommand{\cftfigpresnum}{Code }
% \listoflistings
% \addcontentsline{toc}{chapter}{Codebeispielverzeichnis}

% %===========================================================
% %== Symbolverzeichnis ======================================
% %===========================================================
% \newpage
% %% Symbolverzeichnis
% \makenomenclature
% % This will add the units
% %----------------------------------------------
% \newcommand{\nomunit}[1]{%
% \renewcommand{\nomentryend}{\hspace*{\fill}#1}}
% %----------------------------------------------
% \renewcommand{\nomname}{Symbolverzeichnis}
% %Sorgt dafür, das Symbolverzeichnis im Header steht
% \patchcmd{\thenomenclature}
%   {\chapter*{\nomname}}% usually only \chapter*{\nomname} is issued
%   {\chapter*{\nomname}\markboth{\MakeUppercase\nomname}{\MakeUppercase\nomname}}
%   {}{}
% %-----------------------------------------------------------------------------------
% Symbolverzeichnis
%-----------------------------------------------------------------------------------
% Bsp: \nomenclature[prefix]{symbol}{description}

\nomenclature[]{$c$}{Speed of light in a vacuum inertial system \nomunit{$299,792,458\, m/s$}}

\nomenclature[]{$i$}{Ganzzahlige Laufvariable \nomunit{Integer}}
% \printnomenclature


% %===========================================================
% %== Abkürzungsverzeichnis ==================================
% %===========================================================
% \newpage
% \printglossary[type=\acronymtype, title=Abkürzungsverzeichnis, toctitle=Abkürzungsverzeichnis]
%<<<<<<<<<<<<<<<<<<<<<<<<<<<<<<<<<<<<<<<<<<<<<<<<<<<<<<<<<<<<<<<<<<<<<<<<<<<<<<<<<<<<
% Text Einleitung, Hauptteil usw. 
%<<<<<<<<<<<<<<<<<<<<<<<<<<<<<<<<<<<<<<<<<<<<<<<<<<<<<<<<<<<<<<<<<<<<<<<<<<<<<<<<<<<<

\chapter{Motivation und Problemstellung}
\blindtext

\chapter{Forschungsfrage}
\blindtext


%<<<<<<<<<<<<<<<<<<<<<<<<<<<<<<<<<<<<<<<<<<<<<<<<<<<<<<<<<<<<<<<<<<<<<<<<<<<<<<<<<<<<
% Text Einleitung, Hauptteil usw. 
%<<<<<<<<<<<<<<<<<<<<<<<<<<<<<<<<<<<<<<<<<<<<<<<<<<<<<<<<<<<<<<<<<<<<<<<<<<<<<<<<<<<<
% %===========================================================
% %== Glossar ================================================
% %===========================================================
% %% Glossar
% \renewcommand*{\glossaryentrynumbers}[1]{} %Entfernt die Seitenzahl am Ende der Glossar-Beschreibung

% \printglossary[title=Glossar,toctitle=Glossar]

%===========================================================
%== Literaturverzeichnis ===================================
%===========================================================
% \bibliography{Literatur/literatur}
% \bibliographystyle{alphadin}

%===========================================================
%== Anhang =================================================
%===========================================================
% \appendix
% \chapter{Anhang}
% \newpage
% \section{Ehrenwörtliche Erklärung}
% \input{Anhang/EhrenwörtlicheErklärung.tex}

\end{document}
